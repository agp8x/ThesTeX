% Vorlage für Masterarbeiten
%	Konfiguration: in texmaker unter Benutzer -> Benutzerbefehle einen neuen Befehl anlegen: pdflatex -synctex=1 -interaction=nonstopmode %.tex | bibtex % | makeindex -s config/index.ist % | makeindex %.nlo -s nomencl.ist -o %.nls | pdflatex -synctex=1 -interaction=nonstopmode %.tex | pdflatex -synctex=1 -interaction=nonstopmode %.tex
%	Entsprechende Informationen in den config/metainfo verändern
%   Beispieldokument entsteht bei Kompilierung nach oben angegebener Reihenfolge
\newcommand{\lang}{ngerman}

\documentclass[
   12pt,                % Schriftgroesse 12pt
   a4paper,             % Layout fuer Din A4
   \lang , 			    % Language
   %oneside,            % einseitig
   twoside,             % Layout fuer beidseitigen Druck
   %openright,          % Start auf rechter Seite   
   headinclude,         % Kopfzeile wird Seiten-Layouts mit beruecksichtigt
   headsepline,         % horizontale Linie unter Kolumnentitel
	%footinclude,
	%footexclude,		%fusszeile nicht bercksichtigen
   %plainheadsepline,    % horizontale Linie auch beim plain-Style
   BCOR12mm,            % Korrektur fuer die Bindung
   DIV14,               % DIV-Wert fuer die Erstellung des Satzspiegels, siehe scrguide
   %DIVcalc,							% automatische Berechnung des Satzspiegels
   %openany,             % Kapitel knnen auf geraden und ungeraden Seiten beginnen
   %openright,							% Kapitel auf rechten Seiten beginnen!
   %pointlessnumbers,    % Kapitelnummern immer ohne Punkt
   fleqn,               % fleqn: Glgen links (statt mittig)
   %draft,               % Keine Bilder in der Anzeige, overfull hboxes werden angezeigt
   appendixprefix,				%berschriften des Anhangs +"Anhang"
  %chapterprefix,				%berschriften der Kapitel +"KAPITEL"
  %tablecaptionabove,    %tabellen immer als berschriften
   abstracton,
   pdftex						%berschrift Zusammenfassung einschalten
     ]{scrreprt}

\usepackage{ifthen}
% Formatierungen
\usepackage{typearea}
\areaset[1cm]{16cm}{26cm}
\usepackage[small,bf,up]{caption}  %nicer captions
\renewcommand{\captionfont}{\small}
\usepackage[hyphens]{url}
%\usepackage{picins}
%\usepackage{ae}                 % Für PDF-Erstellung
\usepackage{latexsym}			% schönere Symbole
%\usepackage{booktabs}
\usepackage{color}
\usepackage{float}

% Zeichensätze
\usepackage[utf8]{inputenc}
\usepackage[T1]{fontenc} % T1-kodierte Schriften, korrekte Trennmuster fuer Worte mit Umlauten

% Meta-Informationen
\author{Autor}
\title{Titel (deutsch)}
\subtitle{Title (english)}
\newcommand{\degree}{Abschluss}
\newcommand{\studycourse}{Studiengang}
\newcommand{\advisor}{Betreuer}
\newcommand{\location}{Bamberg}
\subject{\degree arbeit im Studiengang \studycourse\ der Fakultät Wirtschaftsinformatik und Angewandte Informatik der Otto-Friedrich-Universität Bamberg}
\date{xx.xx.xxxx}


\gittrue
\thesistrue


% Sprache
\usepackage[\lang]{babel}
%\usepackage{abstract}

% Mathe und Formeln
\usepackage{calc}
\usepackage[centertags]{amsmath}
\usepackage{amssymb,amsthm,amsfonts}
\usepackage{cancel}  %%druchstreichen von Formeln

% Programmieren mit Latex
\usepackage{ifthen}

% Zeilenabstand
%%   1,1-facher Zeilenabstand   %%
\usepackage{setspace}
\setstretch{1.1}
%\setlength{\parindent}{0em}

\usepackage{dirtree}   %setzen von baumstrukturen
\usepackage{bbding}    % Hände

\usepackage{paralist}      % kleinere Anstände bei enums und itemizes möglich

%%   Fuer anspruchsvolle Tabellen   %%
\usepackage{longtable, colortbl}
\usepackage{multicol, multirow}

%%  Für Grafiken %%
\usepackage[pdftex]{graphicx}
\usepackage{tikz}
\usepackage{pgfplots}
\usetikzlibrary{arrows,shapes,fit,positioning,decorations,backgrounds,shadows}


%%   Fuer Bibtex nach APA Style (American Psychology Association)   %%
\usepackage[numbers]{natbib}

% Seiten im Querformat
\usepackage{lscape}

%%   intoc zur Aufnhame des Abkuerzungs- und Symbolverzeichnisses ins Inhaltsverzeichnis  
\usepackage[intoc]{nomencl}
\setlength{\nomlabelwidth}{.20\hsize}
%\renewcommand{\nomlabel}[1]{#1 \dotfill}
\setlength{\nomitemsep}{-\parsep}
\makenomenclature
\ifthenelse{\equal{\lang}{ngerman}}{%
\renewcommand{\nomname}{Abk\"urzungsverzeichnis}
\newcommand{\nomaltname}{Symbolverzeichnis}
}{%
\renewcommand{\nomname}{List of Abbreviations}
\newcommand{\nomaltname}{List of Symbols}
}%
\newcommand{\nomaltpreamble}{}
\newcommand{\nomaltpostamble}{}
\newcommand{\usetwonomenclatures}{\nomenclature[\switchnomitem]{}{}}
\newcommand{\switchnomitem}{R}
\renewcommand{\nomgroup}[1]{%
\ifthenelse{\equal{#1}{\switchnomitem}}{\switchnomalt}{}}
\newcommand{\switchnomalt}{%
\end{thenomenclature}
\newpage
\renewcommand{\nomname}{\nomaltname}
\renewcommand{\nompreamble}{\nomaltpreamble}
\renewcommand{\nompostamble}{\nomaltpostamble}
\begin{thenomenclature}
}

%% Stichwortverzeichnis
\usepackage{makeidx}
\makeindex

%%   Hervorhebung der Abkuerzungsbuchstaben   %%
\usepackage[normalem]{ulem}
\newcommand{\m}[1]{\uline{#1}}

% Code-Hervorhebung
% Quellcode
\usepackage{verbatim}            % Quellcode einbinden (\verbatiminput) standardpaket
\usepackage{moreverb} 
% PseudoCode
\usepackage[chapter]{algorithm}
\usepackage{algpseudocode}
%\usepackage{algorithmicx}
\floatname{algorithm}{Algorithmus}
\algrenewcommand{\algorithmiccomment}[1]{\hskip1em\textcolor{gray!60}{$\rhd$ #1}}
\ifthenelse{\equal{\lang}{ngerman}}{%
\renewcommand{\listalgorithmname}{Algorithmen}
\def\algorithmautorefname{Algorithmus}
}{%
\renewcommand{\listalgorithmname}{List of Algorithms}
\def\algorithmautorefname{Algorithm}
}

%% Code Highlighting
\definecolor{mygray}{gray}{.75}
\usepackage{listings} 
\lstset{numbers=left, numberstyle=\tiny, numbersep=6pt} 
\lstset{language=Python}
\lstset{classoffset=1, morekeywords={mycontext}, keywordstyle=\color{darkgreen}, classoffset=0, keywordstyle=\color{darkblue}}
\lstset{basicstyle=\small, showstringspaces=false, commentstyle=\color{mygray}, breaklines=true, captionpos=b}
\ifthenelse{\equal{\lang}{ngerman}}{%
\renewcommand{\lstlistingname}{Code-Ausschnitt}
\renewcommand{\lstlistlistingname}{Code-Ausschnitte}
\def\lstlistingautorefname{Code-Ausschnitt}
}{%
\renewcommand{\lstlistingname}{Code Listing}
\renewcommand{\lstlistlistingname}{Code Listings}
\def\lstlistingautorefname{Code Listing}
}
% Schönere Kapitel?
%\renewcommand{\thechapter}{\Roman{chapter}}
\usepackage{titlesec}
\usepackage{fix-cm}
\titleformat{\chapter}[display]
{\bfseries\large} %war \Large
{ %\Huge\textsc{\chaptertitlename}
\hfill\fontsize{100}{70}\selectfont\color{lightgray}\textbf{\thechapter}} %war 120 statt 100
{-4ex} % war -2ex
{\filleft\fontsize{30}{35}\selectfont\scshape} %war 50 statt 30 und 70 statt 45
[\vspace{-2ex}] %war 0ex
\usepackage{gitinfo2}

%%%%%%%%%%%%%%%%%%%%%%%%%%%%%%%%%%%%%%%%%%%%%%%%%%%%%%%%%%%%%%%%%%%%%%%%%%%%%%%%%%%%%%%%%%%
%%                                   TITLE SETUP                                         %%
%%%%%%%%%%%%%%%%%%%%%%%%%%%%%%%%%%%%%%%%%%%%%%%%%%%%%%%%%%%%%%%%%%%%%%%%%%%%%%%%%%%%%%%%%%%
\makeatletter
\usepackage{wallpaper}
\renewcommand{\maketitle}{
\begin{titlepage}
\ThisCenterWallPaper{1}{image/titlepage.pdf}

\setstretch{1.2}
\vspace*{55mm}
\begin{minipage}[t]{2cm}
\textsc{Thema:}
\end{minipage}
\begin{minipage}[t]{12cm}
\textbf{\Large \@title}\\[10mm]
\textbf{\large \@subtitle \normalsize}
\end{minipage}\\[25mm]
\centering
\Huge \textbf{\degree arbeit}\\
\vspace{1cm}
\Large
im Studiengang \studycourse\ der Fakultät Wirtschaftsinformatik und Angewandte Informatik der Otto-Friedrich-Universität Bamberg\\
\normalsize
\vfill
\begin{flushleft}
\begin{tabbing}
xxxxxxxxxxxxxxx\=xxxxxxxxxxxxxx\kill
Verfasser: \> \@author\\
Themensteller:\> \advisor \\
Abgabedatum:\> \@date\\
\end{tabbing}
\end{flushleft}
\end{titlepage}
}
\makeatother


%%%%%%%%%%%%%%%%%%%%%%%%%%%%%%%%%%%%%%%%%%%%%%%%%%%%%%%%%%%%%%%%%%%%%%%%%%%%%%%%%%%%%%%%%%%
%%                                 Kopf und Fusszeilen %%
%%%%%%%%%%%%%%%%%%%%%%%%%%%%%%%%%%%%%%%%%%%%%%%%%%%%%%%%%%%%%%%%%%%%%%%%%%%%%%%%%%%%%%%%%%%

\usepackage[automark]{scrpage2}            % Kopf und Fusszeilen-Layout
\renewcommand{\headfont}{\normalfont\sffamily\itshape}    % Kolumnentitel serifenlos
\renewcommand{\pnumfont}{\normalfont\sffamily}    % Seitennummern serifenlos
\pagestyle{scrheadings}
%\pagestyle{scrplain}
\ihead[]{\headmark}              % Kolumnentitel immer oben innen
\chead[]{}                       % Mitte leer lassen
\ohead[\pagemark]{\pagemark}     % Seitennummern immer oben aussen
%\ohead[]{}
\ofoot[]{}                       % Seitennummern in der Fusszeile loeschen
\cfoot[]{}                       % Seitennummern in der Fusszeile loeschen

%%%%%%%%%%%%%%%%%%%%%%%%%%%%%%%%%%%%%%%%%%%%%%%%%%%%%%%%%%%%%%%%%%%%%%%%%%%%%%%%%%%%%%%%%%%
%%                                 Selbständigkeitserklärung                             %%
%%%%%%%%%%%%%%%%%%%%%%%%%%%%%%%%%%%%%%%%%%%%%%%%%%%%%%%%%%%%%%%%%%%%%%%%%%%%%%%%%%%%%%%%%%%
\makeatletter
\newcommand{\erklaerung}{
\newpage
\section*{Eidesstattliche Erklärung}
\vspace{25mm}

Ich erkläre hiermit gemäß § 17 Abs. 2 APO, dass ich die vorstehende \degree arbeit selbständig verfasst und keine anderen als die angegebenen Quellen und Hilfsmittel benutzt habe.\\[20mm]

\begin{minipage}{0.4\textwidth}
\location , \@date \hfill \\
\textcolor{white}{M} 
\end{minipage}
\begin{minipage}{0.6\textwidth}
\begin{flushright}
\begin{center}
\textcolor{white}{M}\ldots\ldots\ldots\ldots\ldots\ldots\ldots\ldots\ldots\ldots\ldots\ldots\\
\@author \vfill
\end{center}
\end{flushright}
\end{minipage}
\newpage
}
\makeatother
   
% Hyperref
\usepackage{hyperref}
\definecolor{darkblue}{rgb}{0,.05,.54}
\definecolor{darkgreen}{rgb}{0,.54,.05}
\makeatletter
\hypersetup{pdftitle={\@title}, pdfauthor={\@author}, pdfsubject={\@subject}, pdfkeywords={\gitAbbrevHash}, colorlinks=true, breaklinks=true, linkcolor=darkblue, menucolor=darkblue, urlcolor=darkblue, citecolor=darkblue, filecolor=darkblue}
\makeatother
\urlstyle{same}
\usepackage[all]{hypcap}


%%%%%%%%%%%%%%%%%%%%%%%%%%%%%%%%%%%%%%%%%%%%%%%%%%%%%%%%%%%%%%%%%%%%%%%%%%%%%%%%%%%%%%%%%%%
%%                                   COMMAND SETUP                                       %%
%%%%%%%%%%%%%%%%%%%%%%%%%%%%%%%%%%%%%%%%%%%%%%%%%%%%%%%%%%%%%%%%%%%%%%%%%%%%%%%%%%%%%%%%%%%
\newcommand{\HRule}{\rule{\linewidth}{0.5mm}}

%#1 Breite
%#2 Datei (liegt im image Verzeichnis)
%#3 Beschriftung
%#4 Label fuer Referenzierung
\newcommand{\image}[4]{
\begin{figure}[H]
\centering
\includegraphics[width=#1]{image/#2}
\caption{#3}
\label{#4}
\end{figure}
}

%#1 Datei (liegt im graphic Verzeichnis)
%#2 Beschriftung
%#3 Label fuer Referenzierung
\newcommand{\tikzimage}[3]{%
\begin{figure}[H]%
\centering%
\input{graphic/#1.tikz}%
\caption{#2}%
\label{#3}%
\end{figure}
}

%#1 Datei (liegt im graphic Verzeichnis)
%#2 Beschriftung
%#3 Label fuer Referenzierung
%#4 Skalierungsfaktor
\newcommand{\scaletikzimage}[4]{%
\begin{figure}[H]%
\centering%
\scalebox{#4}{%
\input{graphic/#1.tikz}}%
\caption{#2}%
\label{#3}%
\end{figure}
}

%#1 Breite
%#2 Höhe
%#2 Datei (liegt im image Verzeichnis)
%#3 Beschriftung
%#4 Label fuer Referenzierung
\newcommand{\imagebh}[5]{
\begin{figure}[H]
\centering
\includegraphics[width=#1, height=#2]{image/#3}
\caption{#4}
\label{#5}
\end{figure}
}

%#1 Breite
%#2 Datei (liegt im image Verzeichnis)
%#3 zugehörige Bildunterschrift
%#4 Beschriftung
%#5 Label fuer Referenzierung
\newcommand{\mathimage}[5]{
\begin{figure}[H]
\centering
\includegraphics[width=#1]{image/#2}\\
#3
\caption{#4}
\label{#5}
\end{figure}
}

%#1 algorithm name
%#2 algorithm label
%#3 file name in code-folder
\newcommand{\pseudo}[3]{%
\small%
\begin{algorithm}[H]%
\caption{#1}%
\label{#2}%
\input{code/#3.tex}%
\end{algorithm}%
\normalsize%
}

%#1 Text der als todo dargestellt werden soll
\newcommand{\todo}[1]{
\begin{quote}
\textcolor{red}{\textbf{TODO: #1}}
\end{quote}
}

\newcommand{\rimage}[2]{
\begin{figure}[H]
\centering
#1
%\caption{#2}
\end{figure}
}

\newcommand \rack {
       {\LARGE $\square$}
}

\newcommand \deftab
{\hspace{1.5cm}\=abcdfffefghijk\hspace{1cm}\=1\hspace{1.5cm}\=1\hspace{1.5cm}\=1\hspace{1.5cm}\=1\hspace{1.5cm}\=1\hspace{1.5cm}\=asdjadj\kill}

\newcommand \einsbisfuenf
{\> {\bf -2} \> {\bf -1} \> {\bf 0} \> {\bf 1} \> {\bf 2} \>}

% #1 videofile
% #2 scalefactor
\newcommand{\video}[2]{%
\includemovie[text={\includegraphics[scale=#2]{praesi/video/#1.png}}, autoplay, mouse=true, repeat=1]{}{}{praesi/video/#1.swf}}

     
\begin{document}
% 1.1-facher Zeilenabstand für die Titelseite und die Verzeichnisse 
\setstretch{1.1}
%% Titelseite
\maketitle

%% Seitenlayout
\pagestyle{scrplain}
\cleardoubleemptypage 
\pagenumbering{Roman}
\tableofcontents
\newpage
\listoffigures
\newpage
\listoftables
\newpage
\listofalgorithms
\newpage
%%   Falls nur Abkuerzungs- oder Symbolverzeichnis benoetigt wird, folgende Befehle benutzen   %%
%\printnomenclature

%%   Abkuerzungs- und Symbolverzeichnis   %%
%%   Zum alphabetischen Sortieren in der Konsole "makeindex filename.nlo -s nomencl.ist -o filename.nls" ausfuehren; nach dem zweiten Kompilieren erscheint das Abkuerzungsverzeichnis dann im Dokument   %%
%%   Alternativ im TeXnicCenter unter Ausgabe - Ausgabeprofile definieren - Nachbearbeitung einen Postprocessor ("Neu"-Button oben rechts) erstellen. Nach Bennenung unten als Anwendung Makeindex (aus Verzeichnis "./miktex/bin/makeindex.exe") auswaehlen und als Argumente folgenden String setzen: "%bm".nlo -s nomencl.ist -o "%bm".nls"   %%
\usetwonomenclatures
\printnomenclature

\cleardoubleemptypage
\pagestyle{scrheadings}
\pagenumbering{arabic}
\setcounter{page}{1}
% 1.5-facher Zeilenabstand für die Arbeit 
\setstretch{1.5}

%===============================================================================
% LATEX-Dokument: Kapitel laden
%===============================================================================
%
% hier einzelne Kapitel mit \input{Kapitel-File} einfügen
%

\ifthenelse{\equal{\showbuildnumber}{true}}{
    \chapter{VERSION}
    \IfFileExists{build.number}{
        \section{buildnumber}
        \verbatiminput{build.number}
    }{}
    \section{git}
    \#: \gitAbbrevHash\\
    @: \gitAuthorIsoDate\\
    \gitReferences
    \subsection{gitinfo2 -- setup}
    \url{https://www.ctan.org/tex-archive/macros/latex/contrib/gitinfo2}
    \subsubsection{git hooks}
    To fill watermark at buttom, deploy gitinfo2-hook.txt to githooks: (copy and make executable)
    \begin{itemize}
        \item .git/hooks/post-checkout
        \item .git/hooks/post-commit
        \item .git/hooks/post-merge
    \end{itemize}
    \subsubsection{remove watermark}
    To disable watermark, remove option \texttt{[mark]} from \textbackslash usepackage[mark]\{gitinfo2\} in \textit{config/commands.tex} at line 45.
}{}

\chapter{Einleitung}\label{cha:ein}
Einleitung nach \autoref{cha:ein}

\chapter{Hauptteil}\label{cha:haupt}
\section{Bilder und Grafiken}\label{sec:grafiken}
\subsection{Bilder}\label{subsec:bilder}
Bilder befinden sich im Image-Orgner und lassen sich mit \textbackslash image\{Breite\}\{Datei im Image-Verzeichnis\}\{Beschriftung\}\{Label\} einbinden. \image{3cm}{logo.png}{Uni-Logo}{img:uni} Die Referenzierung erfolg mittels \textbackslash autoref\{Label\}, also z.B. \autoref{img:uni}.
\subsection{Grafiken mit TikZ}
Grafiken im TikZ-Framework\footnote{\url{http://www.tn-home.de/TUGDD/Stuff/TikZ_final.pdf}} lassen sich mit dem Befehl \textbackslash scaletikzimage\{Datei im Image Verzeichnis\}\{Beschriftung\}\{Label\}\{Skalierungsfaktor\} einbinden. \scaletikzimage{tikz}{TikZ-Grafik}{img:tikz}{0.9}
\section{Tabellen}
Tabellen lassen sich mit dem Environment\\
\textbackslash begin\{longtable\}[H h t b c]\{Spaltendefinitionen\} ...\\
\qquad\qquad \textbackslash caption\{Tabellenunterschrift\}\\
\qquad\qquad \textbackslash label\{Label\}\\
\textbackslash end\{longtable\}\\
 definieren\footnote{\url{ftp://ftp.dante.de/pub/tex/macros/latex/required/tools/longtable.pdf}}\\
\begin{longtable}[H]{|p{0.2\textwidth}|p{0.2\textwidth}|p{0.2\textwidth}|}
\hline
A&B&C\\
\hline
\caption{Tabelle 1}
\label{tab:tab1}
\end{longtable}
\section{Code-Ausschnitte}
Pseudo-Code Ausschnitte lassen sich mit \textbackslash pseudo\{Name des Algorithmus\}\{Label\}\{Datei im Code-Verzeichnis\} einbinden.
\pseudo{Mittelwert}{lst:mean}{code}
\section{Zitate}
Mit \textbackslash nocite*\{\} lassen sich alle Einträge in der Bibliography ausgeben. Mit \textbackslash cite[S. xx]\{Key\} lassen sich Zitate einfügen. Z.B. \cite[S. 234]{Ferstl.2008} \nocite*{}
\section{Abkürzungen und Symbole}
\subsection{Abkürzungen}
Abkürzungen können mit \textbackslash nomenclature\{Abk\}\{\textbackslash m\{Abk\}ürzung\} \nomenclature{Abk}{\m{Abk}ürzung} angegeben werden. Diese werden alphabetisch sortiert in ein Abkürzungsverzeichnis aufgenommen.
\subsection{Symbole}
Symbole können mit \textbackslash nomenclature[s]\{$E=mc^2$\}\{Energie\}
\nomenclature[s]{$E=mc^2$}{Energie} in das Symbolverzeichnis aufgenommen werden.
\section{Stichwortverzeichnis}
Stichwörter\index{Stichwort} können mit \textbackslash index\{Stichwort\} angelegt werden. Weitere Schlagwörter\index{Stichwort!Schlagwort} hängt man mit \textbackslash index\{Stichwort!Schlagwort\} an \footnote{\url{http://www2.informatik.hu-berlin.de/~piefel/LaTeX-PS/V03-index.pdf}}.


\cleardoubleemptypage 
\pagestyle{scrplain}
% 1.1-facher Zeilenabstand für das Literaturverzeichnis und den Anhang 
\setstretch{1.1}
\bibliographystyle{literatur/IEEEtran}%alphadin}%alpha}%abbrvdin}%natdin}%plaindin}%apa}
%plaindin = Literaturverzeichnis nach deutschem DIN-Standard siehe Webseite von Lorenzen www.haw-hamburg.de/pers/Lorenzen
\bibliography{literatur/bib}

\cleardoubleemptypage% 
% Stichwortverzeichnis soll im Inhaltsverzeichnis auftauchen
% Sprungmarke mit Phantomsection korrigiert
\phantomsection%
\addcontentsline{toc}{chapter}{Index}%
% Stichwortverzeichnis endgueltig anzeigen
\printindex%


\appendix
\setstretch{1.5}
\chapter{Anhang}
\setstretch{1.1}

\cleardoubleemptypage 
\erklaerung

\end{document}
